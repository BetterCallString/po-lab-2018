\documentclass[10pt,handout]{beamer}
%\documentclass[10pt]{beamer}
\usetheme{Boadilla}
\usepackage[utf8]{inputenc}
\usepackage{amsmath}
\usepackage{amsfonts}
\usepackage{amssymb}
\usepackage{tikz}
\usetikzlibrary{arrows,decorations.markings}
\usetikzlibrary{shapes.geometric}
\usetikzlibrary{positioning}
\usepackage[absolute,verbose,overlay]{textpos}
\usepackage{stackengine}
\usepackage{scalerel}
%\usepackage{tikz-uml}
\usepackage{listings}
\usepackage{forest}
\usepackage{booktabs}
\usepackage{listings}
\usepackage{fancyvrb}
\usepackage{pgfplots}

\usepackage{amsmath}
\usepackage{algorithm}
\usepackage[noend]{algpseudocode}
\usepackage{mathtools}

\lstset { %
    language=C++,
    backgroundcolor=\color{blue!10}, % set backgroundcolor
    basicstyle=\footnotesize,% basic font setting,
        commentstyle=\color[rgb]{0.026,0.112,0.095},
                basicstyle=\ttfamily,
                keywordstyle=\color{blue}\ttfamily,
                stringstyle=\color{red}\ttfamily,
                commentstyle=\color{gray}\ttfamily,
                morecomment=[l][\color{magenta}]{\#}
}

\graphicspath{{figures/}}


\newlength\triwidth
\author{Alex, Jakob, Julian, Michael, Peter}
\title{Value Set Analysis in LLVM}
%\setbeamercovered{transparent} 
%\setbeamertemplate{navigation symbols}{} 
%\logo{} 
\institute{IN2053 - Program Optimization Lab 2018} 
%\date{February 8, 2017} 
%\subject{} 

\setbeamertemplate{footline}[frame number]{}

%gets rid of bottom navigation symbols
\setbeamertemplate{navigation symbols}{}
\expandafter\def\expandafter\insertshorttitle\expandafter{%
  \insertshorttitle\hfill%
  \insertframenumber\,/\,\inserttotalframenumber}

%gets rid of footer
%will override 'frame number' instruction above
%comment out to revert to previous/default definitions
%\setbeamertemplate{footline}{}

\setbeamertemplate{bibliography item}{}



\bibliographystyle{apalike}
% make bibliography entries smaller
%\renewcommand\bibfont{\scriptsize}
% If you have more than one page of references, you want to tell beamer
% to put the continuation section label from the second slide onwards
\setbeamertemplate{frametitle continuation}[from second]
% Now get rid of all the colours
\setbeamercolor*{bibliography entry title}{fg=black}
\setbeamercolor*{bibliography entry author}{fg=black}
\setbeamercolor*{bibliography entry location}{fg=black}
\setbeamercolor*{bibliography entry note}{fg=black}

\begin{document}

\begin{frame}
\titlepage
\end{frame}

%\begin{frame}
%\tableofcontents
%\end{frame}

\newcommand{\linen}[2]{ &&\qquad #1 & #2 \\}
\newcommand{\linenn}[4]{#1 & #2 &\qquad #3 & #4 \\}
\newcommand{\linel}[1]{ &&#1 \\}

\usetikzlibrary{decorations.text}
\definecolor{mygray}{RGB}{208,208,208}
\definecolor{mymagenta}{RGB}{226,0,116}
\newcommand*{\mytextstylee}{\sffamily\bfseries\color{white!85}}
\newcommand*{\mytextstyle}{\sffamily\bfseries\color{black!85}}
\newcommand{\arcarrow}[3]{%
   % inner radius, middle radius, outer radius, start angle,
   % end angle, tip protusion angle, options, text
   \pgfmathsetmacro{\rin}{2.5}
   \pgfmathsetmacro{\rmid}{3.0}
   \pgfmathsetmacro{\rout}{3.5}
   \pgfmathsetmacro{\astart}{#1}
   \pgfmathsetmacro{\aend}{#2}
   \pgfmathsetmacro{\atip}{5}
   \fill[mygray, very thick] (\astart+\atip:\rin)
                         arc (\astart+\atip:\aend:\rin)
      -- (\aend-\atip:\rmid)
      -- (\aend:\rout)   arc (\aend:\astart+\atip:\rout)
      -- (\astart:\rmid) -- cycle;
   \path[
      decoration = {
         text along path,
         text = {|\mytextstyle|#3},
         text align = {align = center},
         raise = -1.0ex
      },
      decorate
   ](\astart+\atip:\rmid) arc (\astart+\atip:\aend+\atip:\rmid);
}
\newcommand{\arcarroww}[3]{%
   % inner radius, middle radius, outer radius, start angle,
   % end angle, tip protusion angle, options, text
   \pgfmathsetmacro{\rin}{2.5}
   \pgfmathsetmacro{\rmid}{3.0}
   \pgfmathsetmacro{\rout}{3.5}
   \pgfmathsetmacro{\astart}{#1}
   \pgfmathsetmacro{\aend}{#2}
   \pgfmathsetmacro{\atip}{5}
   \fill[blue, very thick] (\astart+\atip:\rin)
                         arc (\astart+\atip:\aend:\rin)
      -- (\aend-\atip:\rmid)
      -- (\aend:\rout)   arc (\aend:\astart+\atip:\rout)
      -- (\astart:\rmid) -- cycle;
   \path[
      decoration = {
         text along path,
         text = {|\mytextstylee|#3},
         text align = {align = center},
         raise = -1.0ex
      },
      decorate
   ](\astart+\atip:\rmid) arc (\astart+\atip:\aend+\atip:\rmid);
}


\begin{frame}[fragile]{\textcolor{blue}{Fixpoint Algorithm}}
\begin{algorithm}[H]
\caption{Fixpoint Algorithm}
\begin{algorithmic}[1]
\Procedure{Fixpoint}{Function $\mathcal{F}$}
\State $\mathcal{W}$.push($\mathcal{F}$.front())\Comment{get first basic block}
\State \textbf{while} $!\, \mathcal{W}.empty()$ \textbf{then}:
\State \qquad visit $\mathcal{W}$.pop()
\EndProcedure
\end{algorithmic}
\end{algorithm}
\end{frame}



\begin{frame}[fragile]{\textcolor{blue}{Visitor: Entering Basic Block}}
\begin{algorithm}[H]
\caption{Enter basic block BB}
\begin{algorithmic}[1]
\Procedure{Visit}{BB}%\Comment{T: syntax tree}
\State $\mathcal{N}_{BB}= \bigsqcup \left\lbrace \mathcal{N}_{XX} \oplus \mathcal{C}_{XX}(BB) \,|\, XX \in \text{prev}(BB) \right\rbrace$
\State \textbf{for each} $instruction \in \text{insturcions}(BB)$:
\State \qquad visit $instruction$
\EndProcedure
\end{algorithmic}
\end{algorithm}
\vfill
\underline{Implementation details:}
\begin{itemize}
\item Construct $\mathcal{N}_{BB}$ iteratively: $\mathcal{N}_{BB}^{i+1} \gets \mathcal{N}_{BB}^{i} \,\sqcup\, \{ ... \}$ with $\mathcal{N}_{BB}^{0}\gets\perp$
\item if $\{...\}=\perp$: skip it
\item three cases: 
$\perp \sqcup\, x = x$,
$x \,\sqcup \perp = x$, and
$x \,\sqcup\, y$
\end{itemize}
\end{frame}


\begin{frame}[fragile]{\textcolor{blue}{Visitor: Leaving Basic Block (1)}}
\begin{algorithm}[H]
\caption{Visit terminator}
\begin{algorithmic}[1]
\Procedure{Visit}{Terminator}%\Comment{T: syntax tree}
\State $\mathcal{N}_{BB} \gets \mathcal{O}_{BB} \,\sqcup\, \mathcal{N}_{BB}$
\State \textbf{if} $\mathcal{N}_{BB} \sqsubseteq O_{BB}$ \textbf{then}:
\State \qquad \textbf{return}
\Comment state has not changed
\State $\mathcal{O}_{BB} \gets \mathcal{N}_{BB}$
\State \textbf{for each} $block \in \text{next}(BB)$:
\State \qquad\textbf{if} reachable($BB,\,block$) \textbf{then}:
\Comment always true if not-cond. branch
\State \qquad\qquad $\mathcal{W}$.push($block$)
\EndProcedure
\end{algorithmic}
\end{algorithm}
\end{frame}


\begin{frame}[fragile]{\textcolor{blue}{Visitor: Leaving Basic Block (2)}}


\begin{algorithm}[H]
\caption{Visit conditional branch (terminator)}
\begin{algorithmic}[1]
\Procedure{Visit}{Br($x$ \textsc{op} y)}%\Comment{T: syntax tree}
\State $\mathcal{C}_{BB} \gets \emptyset$
\State $\mathcal{C}_{BB} \gets  \mathcal{C}_{BB} \oplus \left\lbrace XX \,\rightarrow\, x\,\rightarrow\, \mathcal{N}_{BB}[x]\,\textsc{op}^\#\,\mathcal{N}_{BB}[y]
\,|\,
XX = \text{next}_1(BB)
\right\rbrace$
\State $\mathcal{C}_{BB}\gets \mathcal{C}_{BB} \oplus ...$
%\Comment update conditions
\State \textsc{VisitTerminator}(Br)
\EndProcedure
\end{algorithmic}
\end{algorithm}


\begin{algorithm}[H]
\caption{Reachability of basic block $BBN$ from $BB$}
\begin{algorithmic}[1]
\Procedure{Reachable}{BB, BBN}%\Comment{T: syntax tree}
\State \textbf{return} $\exists (\#\,\rightarrow\,\perp)\in \mathcal{C}_{BB}[BBN]$
\EndProcedure
\end{algorithmic}
\end{algorithm}
\end{frame}


%\begin{frame}[fragile]{\textcolor{blue}{Visitor (2)}}
%
%\begin{algorithm}[H]
%\caption{Specialized for Nodes}
%\begin{algorithmic}[1]
%\Procedure{Visit}{pair}%\Comment{T: syntax tree}
%\State $t \gets  \mathcal{D}[p.first] \,\sqcup \, p.second$
%%\State $T,\,L\gets$ annotate $T$ and identify loops
%\State \textbf{if} $ t \neq \mathcal{D}[p.first]$ \textbf{then}:
%\State \qquad $\mathcal{D}[p.first] = t$
%\State \qquad\textbf{for each} $e \in p.first.E_{out}$:
%\State \qquad\qquad $\mathcal{W}$.queue(pair($e$, $t$))
%\EndProcedure
%\end{algorithmic}
%\end{algorithm}
%
%\begin{algorithm}[H]
%\caption{Specialized for Edges}
%\begin{algorithmic}[1]
%\Procedure{Visit}{pair}%\Comment{T: syntax tree}
%\State $\mathcal{W}$.queue(pair($p.first.dest$, eval($p.first$ $p.second$)))
%\EndProcedure
%\end{algorithmic}
%\end{algorithm}
%
%\end{frame}

\begin{frame}[fragile]{\textcolor{blue}{Visitor: Binary Expressions}}

\begin{algorithm}[H]
\caption{Phi-Node in basic block BB}
\begin{algorithmic}[1]
\Procedure{Phi}{$x\gets cond\,?\,y\,:\,z$}%\Comment{T: syntax tree}
\State $\mathcal{N}_{BB}\gets\mathcal{N}_{BB}\oplus\left\lbrace x\,\rightarrow\, \mathcal{N}_{BB}[y] \,\sqcup\, \mathcal{N}_{BB}[z] \right\rbrace$
\EndProcedure
\end{algorithmic}
\end{algorithm}

\begin{algorithm}[H]
\caption{Addition in basic block BB}
\begin{algorithmic}[1]
\Procedure{Add}{$x\gets y + z$}%\Comment{T: syntax tree}
\State $\mathcal{N}_{BB}\gets\mathcal{N}_{BB}\oplus\left\lbrace x\,\rightarrow\, \mathcal{N}_{BB}[y]+^\# \mathcal{N}_{BB}[z] \right\rbrace$
\EndProcedure
\end{algorithmic}
\end{algorithm}


\begin{algorithm}[H]
\caption{Multiplication in basic block BB}
\begin{algorithmic}[1]
\Procedure{Mul}{$x\gets y \cdot z$}%\Comment{T: syntax tree}
\State $\mathcal{N}_{BB}\gets\mathcal{N}_{BB}\oplus\left\lbrace x\,\rightarrow\, \mathcal{N}_{BB}[y]\cdot^\# \mathcal{N}_{BB}[z] \right\rbrace$
\EndProcedure
\end{algorithmic}
\end{algorithm}

\end{frame}

%\begin{frame}[fragile]{\textcolor{blue}{Visitor: Binary Expressions}}
%
%
%
%
%\begin{algorithm}[H]
%\caption{Enter basic block BB}
%\begin{algorithmic}[1]
%\Procedure{TERMINATOR}{}%\Comment{T: syntax tree}
%\State $\mathcal{O}_{BB}= \mathcal{O}_{BB} \,\sqcup\, \mathcal{N}_{BB}$
%\EndProcedure
%\end{algorithmic}
%\end{algorithm}
%
%
%\begin{algorithm}[H]
%\caption{Enter basic block BB}
%\begin{algorithmic}[1]
%\Procedure{BR}{}%\Comment{T: syntax tree}
%\State $\mathcal{C}_{BB} = \left\lbrace XX \,\rightarrow\, X\,\rightarrow\, x
%\,|\,
%XX\in \text{next}(BB)
%\right\rbrace$
%\EndProcedure
%\end{algorithmic}
%\end{algorithm}
%
%\end{frame}



\end{document}